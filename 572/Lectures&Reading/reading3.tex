\documentclass[11pt]{article}
\usepackage{url}
\usepackage{pgfplots} 
\pgfplotsset{compat=newest} 
\setlength\topmargin{-0.6cm}   
\setlength\textheight{23.4cm}
\setlength\textwidth{17.0cm}
\setlength\oddsidemargin{0cm} 
\begin{document}
\title{Ling 572 Reading 3}
\author{Daniel Campos  \tt {dacampos@uw.edu}}
\date{02/28/2019}
\maketitle 
\section{ Q1: What does training data look like? That is, a classifier is trained with (x, y) pairs. For this reranking problem,what is x and what is y?   }

\section{ Q2: What happens at the test time? That is, what formula(s) one needs to calculate in order to determine the correct ranking of the candidate parse trees?}

\section{ Q3:Conceptually, a parse tree is representedas a feature vector. What are the features?What are the feature values? How manyfeatures are there?}

\section{Q4: In practice, is it necessary to represent a parse tree as a feature vector? Why or why not?}
\end{document}



