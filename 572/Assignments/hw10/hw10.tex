\documentclass[11pt]{article}
\usepackage{pgfplots}
\usepackage{amssymb}
\usepackage{url}
\pgfplotsset{compat=newest} 
\setlength\topmargin{-0.6cm}   
\setlength\textheight{23.4cm}
\setlength\textwidth{17.0cm}
\setlength\oddsidemargin{0cm} 
\begin{document}
\title{Ling 572 HW10}
\author{Daniel Campos  \tt {dacampos@uw.edu}}
\date{03/20/2019}
\maketitle 
\section{Q1}         
\begin{table}[h]
\centering
\caption{Classification accuracy with {\bf sigmoid} activation function}
\label{table1}
  \begin{tabular}{|c|r|l|l|l|l|l|} \hline
    Expt & \# of          & \# of neurons in & \# of    & mini-batch & test  & CPU time   \\  
    id   &  hidden layer  & hidden layers    & epoches & size       & accuracy  & (in minutes) \\ \hline

    1   &  1              &  30              &  30     &  10       & 76.67     & 0.33 \\ \hline
    2   &  1              &  30              &  30     &  50       &  57.62   & 0.42 \\ \hline
    3   &  1              &  30              &  100    &  10       &  77.90   & 1.2 \\ \hline
    4   &  1              &  60              &  30     &  10       &   71.62  & 1.33\\ \hline
    5   &  2              &  30, 30          &  30     &  10       &  74.19   & 0.5 \\ \hline

    6   &  2              &  40, 20          &  30     &  10       &  76.00   & 0.5\\ \hline
    
    7   &  3              &  20, 20, 20      &  30     &  10       & 73.90    & 0.46 \\ \hline
  \end{tabular}
\end{table}
\section{Q2}
For Q2 I changed ... and ... in file .. lines ...
\begin{table}[h]
\centering
\caption{Classification accuracy with {\bf tanh} activation function}
\label{table1}
  \begin{tabular}{|c|r|l|l|l|l|l|} \hline
    Expt & \# of          & \# of neurons in & \# of    & mini-batch & test  & CPU time   \\  
    id   &  hidden layer  & hidden layers    & epoches & size       & accuracy  & (in minutes) \\ \hline

    1   &  1              &  30              &  30     &  10       &     & \\ \hline
    2   &  1              &  30              &  30     &  50       &     & \\ \hline
    3   &  1              &  30              &  100    &  10       &     &  \\ \hline
    4   &  1              &  60              &  30     &  10       &     & \\ \hline
    5   &  2              &  30, 30          &  30     &  10       &     & \\ \hline

    6   &  2              &  40, 20          &  30     &  10       &     & \\ \hline
    
    7   &  3              &  20, 20, 20      &  30     &  10       &     & \\ \hline
  \end{tabular}
\end{table}
\end{document}

