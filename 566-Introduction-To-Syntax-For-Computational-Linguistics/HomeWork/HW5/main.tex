\documentclass{article}
\usepackage[utf8]{inputenc}
\usepackage{pgfplots,multicol}
\usepackage{tikz-qtree}
\usepackage{url}
\usepackage{hyperref}
\usepackage{xcolor}
\usepackage{avm}
\usepackage{rtrees}
\usepackage{forest}
\useforestlibrary{linguistics}
\newcommand{\comment}[1]{}
\forestapplylibrarydefaults{linguistics}
\hypersetup{
  colorlinks   = true, %Colours links instead of ugly boxes
  urlcolor     = red, %Colour for external hyperlinks
  linkcolor    = blue, %Colour of internal links
  citecolor   = blue %Colour of citations
}
\pgfplotsset{compat=newest} 
\mathchardef\period=\mathcode`.
\newcommand{\textarray}[1]{\ensuremath{\left[ \mbox{\ttfamily\begin{tabular}{l} #1 \end{tabular}}\right]}}
\begin{document}
\title{566 HW5}
\author{Daniel Campos  \tt {dacampos@uw.edu}}
\date{11/1/2019}
\maketitle 
\section{Chapter 8, Problem 5}
\subsection{Kim left and but or or and Sandy stayed}
No.
\subsection{Why}
This is unlicensed since in out updated coordination rule in chapter 8 all items but the main coordinator must have the same form. If we treat the final 'and' as our coordinator then the phrases 'Kim Left' and 'Sandy stayed' agree with each other in their form but disagree with all the extra conjunctions like 'but' and 'or. In other words since all the items but the coordinator do not agree, our grammar will not license the mentioned phrase.
\section{Chapter 8, Problem 6}
\subsection{Write Head-Specifier and Head-Complement Rules for Japanese}
The Head-Specifier does not seem to need to change because items like the determiners are still attaching to the nouns they specify. Based on that the same head-specifier that we have in chapter 8 holds. Written out below. \\
\begin{avm}
\[{\it phrase} \\ SYN & \[ VAL & \[SPR & \< \avml\hfil \avmr  \> \] \] \] \rightarrow {\@1} \textbf{H} \[ SYN \[VAL & \[SPR & \< \avml {\@1} \avmr  \>  \\ COMPS & \< \avml\hfil \avmr  \>    \] \] \]
\end{avm}
\\The Head-Complement needs to get modified since compliments attach to heads slightly differently. Based on our examples, it appears that in Japanese, the head in a complimented phrases is at the end of the sentence. Thus we modify the rule to what we have below, head at the end. \\
\begin{avm}
\[{\it phrase} \\SYN & \[VAL & \[COMPS & \< \avml\hfil \avmr  \> \] \]\] \rightarrow {\@1},...,{\@n} \textbf{H} \[SYN & \[VAL & \[{\it word} \\ COMPS & \< \avml {\@1},...,{\@n} \avmr  \>  \] \] \] \hfil 
\end{avm}
\subsection{Lexical entry for verbs in examples i-iv}
\begin{avm}
\< yonda , \[ {\it word} \\ SYN & \[ HEAD & \[ {\it verb}  \\ FORM & fin \] \\ VAL & \[SPR & \< \avml {\@1} \avmr \> \\ COMPS & \< \avml {\@2} \avmr \> \] \] \\ SEM & \[ MODE & {\it prop} \\ INDEX & {\it s}_1  \\ RESTR \< \avml\hfil \[ RELN  & {\it read} \\ SIT & {\it s}_1  \\ READ & j \\ READER & i \] \avmr \> \] \\ ARG-ST & \< \avml {\@1}NP\[ CASE & {\it nom} \]_i \, {\@2}NP\[CASE & {\it acc}\]_j \avmr \> \] \>
\end{avm} \\
\begin{avm}
\< ageta , \[ {\it word} \\ SYN & \[ HEAD & \[ {\it verb}  \\ FORM & fin \] \\ VAL & \[SPR & \< \avml {\@1} \avmr \> \\ COMPS & \< \avml {\@2} , {\@3} \avmr \> \] \] \\ SEM & \[ MODE & {\it prop} \\ INDEX & {\it s}_1  \\ RESTR \< \avml\hfil \[ RELN  & {\it give} \\ SIT & {\it s}_1  \\ giver & i \\ receiver & j \\ item & k\] \avmr \> \] \\ ARG-ST & \< \avml {\@1}NP\[ CASE & {\it nom} \]_i \, {\@2}NP\[CASE & {\it dat} \]_j, {\@3}NP\[CASE & {\it acc}\]_k \avmr \> \] \>
\end{avm} \\
\begin{avm}
\< kita , \[ {\it word} \\ SYN & \[ HEAD & \[ {\it verb}  \\ FORM & fin \] \\ VAL & \[SPR & \< \avml {\@1} \avmr \> \\ COMPS & \< \avml \hfil \avmr \> \] \] \\ SEM & \[ MODE & {\it prop} \\ INDEX & {\it s}_1  \\ RESTR \< \avml\hfil \[ RELN  & {\it arrive} \\ SIT & {\it s}_1  \\ ARRIVER & i \] \avmr \> \] \\ ARG-ST & \< \avml {\@1}NP\[ CASE & {\it nom} \]_i  \avmr \> \] \>
\end{avm}
\subsection{Lexical entries for the nouns Taroo and hon.}
\begin{avm}
\< Taroo , \[ SYN & \[ HEAD & \[{\it noun} \\ AGR & \[NUM & sg \\ PER & 3rd \\CASE & {\it dat}\] \\ VAL & \[ SPR & \< \avml\hfil \avmr \> \\ COMPS & \< \avml\hfil \avmr \>  \] \] \]  \\ SEM & \[INDEX & i \\ RESTR & \< \avml\hfil \[ RELN  & name \\ NAME & Taroo \\ NAMED & i \] \avmr \> \] \]\>
\end{avm} \\
\begin{avm}
\< hon , \[ SYN & \[ HEAD & \[{\it noun} \\ AGR & \[NUM & sg \\ PER & 3rd \\CASE & {\it acc}\] \\ VAL & \[ SPR & \< \avml\hfil \avmr \> \\ COMPS & \< \avml\hfil \avmr \>  \] \] \] \\ SEM & \[INDEX & i \\ RESTR & \< \avml\hfil \[ RELN  & book \\ INST & i \] \avmr \> \] \]\>
\end{avm}
\subsection{Lexical rule for deriving the inflected forms ending in -o from the
nominal lexemes.}
For my rule which I am calling the accusative nominal rule I have a helper function $F_{ACC}$  which makes the have the -o ending.
\\{\it Accusative nominal rule.}\\
\begin{avm}
\[{\it i-rule} \\ INPUT & \< \avml{\@1},{\it cntn-lxm} \avmr \> \\ OUTPUT & \< \avml F_{ACC}({\@1}) , \[{\it cntn-lxm} \\ SYN & \[HEAD &\[AGR & \[CASE & {\it acc}\] \] \] \] \avmr \> \]
\end{avm}
\section{Chapter 8, Problem 7}
\subsection{What is the case of the CAUSER argument in iii}
Nom
\subsection{What is the case of the CAUSEE argument in iii}
Dat
\subsection{Lexical Entry tabesaseta}
\begin{avm}
\< tabesaseta , \[ {\it word} \\ SYN & \[ HEAD & \[ {\it verb} \\ FORM & fin \] \\ VAL & \[SPR & \< \avml {\@1} \avmr \> \\ COMPS & \< \avml {\@2} , {\@3} \avmr \> \] \] \\ ARG-ST & \< \avml {\@1}NP\[CASE & {\it nom} \]_k , {\@2}NP\[CASE & {\it dat}\]_i , {\@3}NP\[CASE & {\it acc}\]_j \avmr \> \\ SEM & \[ MODE & {\it prop} \\ INDEX & {\it s}_2  \\ RESTR \< \avml \[ RELN  & {\it eat} \\ SIT & {\it s}_1 \\ EATER & i \\CAUSER & k \\ MEAL & j \] , \[ RELN  & {\it cause} \\ SIT & {\it s}_2 \\ CAUSEE & i \\CAUSER & k \\ CAUSED-EVENT & s_1 \] \avmr \> \] \] \>
\end{avm}
\subsection{Causative Lexical Rule for Japanese}
For my rule I have a helper function $F_{CLR}$ which modifies our verb to become a causative verb.\\
\begin{avm}
\[{\it d-rule} \\  INPUT & \< \avml{\@1}, \[{\it verb-lxm} \\ SYM & \[VAL & \[SPR & \< \avml {\@2} \avmr \> \\ COMPS & \< \avml {\@3} \avmr \> \] \\ SEM & \[RESTR & \< \avml {\@A} \avmr \> \]   \\ ARG-ST & \< \avml {\@2},{\@3} \avmr \> \] \]  \avmr \> \\ OUTPUT & \< \avml F_{CLR}({\@1}) , \[{\it verb-lxm} \\ SYM & \[VAL & \[SPR & \< \avml {\@2} \avmr \> \\ COMPS & \< \avml {\@3} \oplus ... \avmr \> \] \\ SEM & \[RESTR & \< \avml {\@A} \oplus ... \avmr \> \]   \\ ARG-ST & \< \avml {\@2},{\@3} \oplus, ... \avmr \> \] \]  \avmr \> \]
\end{avm}
\end{document}
